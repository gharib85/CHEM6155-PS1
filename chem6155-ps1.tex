\documentclass{chem6155problemset}
\usepackage{chemfig}
\usepackage{enumitem}

\title{CHEM6155 -- Problem Set 1\\Math Foundations, Isotopes, Mathematica}
\date{DUE: FRIDAY of Week 3}



\begin{document}

\maketitle

\section{Mathematical Foundations}
The following problems are roughly aligned with the indicated chapters of Steiner, The Chemistry
Maths Book, 2nd Edition, Oxford University Press 2008. If you have difficulty with any
of these problems, please read the corresponding chapter. If that is not sufficient
to help you solve it, please contact one of the module instructors for help. Some of the problems
require you to use \emph{Mathematica}, which is available to Students from the University. 
Install a copy of it on your computer. To get started, there are many tutorials and videos
available online.

\paragraph{1.1 Functions, Differentiation, and Integration}
\begin{enumerate}
 	\item Define the following properties of functions $y=f(x)$, and provide an example for each. You
	may have to look up the definitions; if so, provide the source.
	  \begin{itemize}
	  	\item monotonous
		\item bounded
		\item divergent
		\item continuous
		\item differentiable
		\item periodic
	  \end{itemize}
	  
	 \item Indicate which of the above properties the following functions satisfy (state
	 your reasoning):
	 	\begin{itemize}
			\item $f(x)=x^2\,e^{-x}$
			\item $f(x)=\tan(1/x)$
			\item $f(x)=\cos\frac{2\pi x}{L} + 2 \sin\frac{6\pi x}{L}$
			\item $f(x) = x^3-4x^2-11x+30$
			\item $f(x) = \dfrac{x^3-4x^2-11x+30}{x^3-6}$
		\end{itemize}
		
	\item Plot the above functions using \emph{Mathematica}. Indicate the approximate 
	positions of roots, stationary points and poles in the plots.
	
	\item Find the exact locations of the stationary points in these functions. 
	Try to verify your answers using \emph{Mathematica}. You may need to use
	the derivative function \verb@D[f[x],x]@ and the equation solver \verb@Solve[f[x]==0,x]@
	 
	 
	\item Evaluate the following definite integrals (show your working):
	 \begin{itemize}
	 	\item $\displaystyle\int\limits_0^{\pi^2} \frac{\sin(\sqrt{x}+\pi)}{\sqrt{x}}\,\mathrm{d}x$
		\item $\displaystyle\int\limits_0^\infty x e^{-x^2}\,\mathrm{d}x$
	 \end{itemize}
	 
	 \item Plot the above integrands with \emph{Mathematica}, and verify your integration
	 results by using \newline \verb@Integrate[f[x],{x,a,b}]@.
	 
	 \item Two functions $f(x)$ and $g(x)$ are inverses of each other, in the sense that
	 $f(g(x))=x$. Show that 
	 \[ g'(x) = [f'(g(x))]^{-1}. \]
	 
	 \item Show that $\displaystyle\frac{\mathrm{d}}{\mathrm d x} \tan x = 1+\tan^2 x.$
	 
	 \item Show that $\displaystyle\frac{\mathrm{d}}{\mathrm d x} \arctan x = \frac{1}{1+x^2}$.
	 
	 \item In NMR spectroscopy, resonance lines often take the shape of the Lorentzian function
	 \[\displaystyle g(\omega) = \frac{1}{\pi} \frac{T}{1+T^2(\omega-\omega_0)^2}\]
	 \begin{itemize}
	  \item Plot this function for various choices of the parameters $T$ and $\omega_0$. What
	 	role do these parameters play? 
	  \item Find the integral $\int_{-\infty}^\infty\,g(\omega)\,\mathrm{d}\omega$ as a 
	  function of the parameters $\omega_0$ and $T$. Comment on your finding.
	  \end{itemize}
\end{enumerate}

\paragraph{1.2 Algebraic Equations and Complex Numbers}

\begin{enumerate}[resume]
	\item Provide brief explanations for the following terms and expressions ($z$ is a complex number)
	\begin{itemize}
		\item complex conjugate, $z^\ast$
		\item modulus, $|z|$
		\item argument, $\arg z$
	\end{itemize}
	
	\item find all complex solutions to the following equations:
	\begin{itemize}
	 	\item $z^5 + 1 = 0$
		\item $z^3+2z^2+3z = 0$
		\item $\cos\frac{2\pi z}{L} + \frac{i}{2} =0$
	\end{itemize}
\end{enumerate}

\paragraph{1.3 Vectors and Vector Fields}
\begin{enumerate}[resume]
  \item The two vectors $\mathbf{a}=(1,-2,0)^T$ and $\mathbf{b}=(2,1,-1)^T$ span 
    plane a $P$ in 3D space, which contains the point $(1,0,0)$. Find
    the equation that describes the plane in the form $Ax+By+Cz+D=0$, i.e., determine
    the coefficients $A,B,C$, and $D$. 
\end{enumerate}

\paragraph{1.4 Matrices and Eigenvalue Problems}

\section{Isotopes and Larmor Precession}
\paragraph{Isotopomers of Ethanol}
	Ethanol (\chemfig{CH_3CH_2OH}) consists of the elements carbon, oxygen, and hydrogen.
	\begin{enumerate}
	\item How many distinct stable (non-radioactive) isotopomers of ethanol exist? 
	\item Create a table of all hydrogen isotopomers of ethanol, and indicate 
	their natural abundance to two significant figures. Are any of these chiral?
	\item How would the numbers in the above table change if the natural abundance
	of deuterium were 50\% instead of 0.015\%?
	\end{enumerate}

	

\end{document}

